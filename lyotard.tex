\documentclass{scrartcl}

\usepackage{amssymb}
\usepackage{amsmath}
\usepackage{tikz}

%Lyotard, J.F.; Beakley, B. (trans.). (1991 [1986]). Phenomenology. Albany, NY: SUNY Press, pp. 114-5
%diagram from p. 115, fig. 2

\begin{document}
	
	\hspace{2cm}
	\scalebox{0.8}{
	\begin{tikzpicture}
	\draw[thick] (-5.5,0)--(5.5,0);			%x-axis
	\draw[thick] (-2.75,-4.5)--(-2.75,0);	%leftmost
	\draw[thick,dashed] (0,3.5)--(0,0); 	%mid-up
	\draw[thick] (0,-3.5)--(0,0);			%mid-down
	\draw[thick,dashed] (2.75,4.5)--(2.75,0);%rightmost
	
	%diagonal lines
	%\draw[thick,dashed] (-5.8,-2.25)--(3.25,4.5);
		\draw[thick,dashed] (-2.75,0)--(3.25,4.5);
		\draw[thick] (-5.8,-2.25)--(-2.75,0);
	%\draw[thick,dashed] (-4.8,-3.6)--(4.25,3.15);
		\draw[thick,dashed] (0,0)--(4.25,3.15);
		\draw[thick] (-4.8,-3.6)--(0,0);
	%\draw[thick,dashed] (-3.8,-4.85)--(5.25,1.9);
		\draw[thick,dashed] (2.75,0)--(5.25,1.9);
		\draw[thick] (-3.8,-4.85)--(2.75,0); 
		
	%labels
	\node at (0.35,1.9)   {\small $B_1$};
	\node at (3.1,1.9)    {\small $C_1$};
	\node at (3.1,3.9)    {\small $C_2$};
	\node at (-5.15,0.25) {\small Past};
	\node at (5,0.25) 	  {\small Future};
	\node at (-3,0.2)     {\small $A$};
	\node at (-3,-1.9)    {\small $A'$};
	\node at (-3,-3.9) 	  {\small $A''$};
	\node at (-0.25,0.2)  {\small $B$};
	\node at (-0.25,-1.9) {\small $B'$};
	\node at (2.5,0.2)	  {\small $C$};
	\end{tikzpicture}
		} %for scalebox
	
	\vspace{0.5cm}
	
	{\small Ultimately, then, in what does the temporality of consciousness consist? [...] Merleau-Ponty (in \textit{PP}, 477 [417]) borrows the schema below from Husserl (\textit{TC}, \S10), where the horizontal line represents the series of nows, the oblique lines the profiles of these same nows viewed from a later now, and the vertical lines the successive profiles of the same now.}
		
	{\small ``Time is not a line, but a network of intentionalities.'' When I slide from $A$ to $B$, I keep hold of $A$ throughout $A'$ and beyond. We might say that the problem has only been pushed back a step: since it amounts to explaining the units of the flux of experiences, we must here establish the vertical unity of $A'$ with $A$, then of $A''$ with $A'$ and $A$, etc. The question of the unity of $B$ with $A$ is replaced by that of the unity of $A'$ with $A$. This is where Merleau-Ponty, following Husserl and Heidegger, established a fundamental distinction concerning our problem of the historian's consciousness: in the \textit{purposive} memory and the \textit{voluntary} evocation of a distant past, there is a place for the syntheses of identification which allow me, for example, to connect \textit{this} joy to its time of provenance, that is, to localize it. But this intellectual operation, performed by the historian, itself presupposes a natural and primordial unity by which it is $A$ itself that I reach in $A'$. It might be said that $A$ is altered in $A'$, and that memory transforms its object|a rather banal proposition in psychology. To which Husserl responds that this scepticism, lying at the base of historicism, undercuts itself as scepticism, since alteration implies \textit{that in some way we know the thing altered}|that is, $A$ itself. Thus there is a \textit{passive synthesis} of $A$ with its perspectival shadings|it being understood that this term does not explain the temporal unity, but allows us at least to pose the problem correctly.}
	
	{\small We must still note that when $B$ becomes $C$, $B$ also becomes $B'$, and that simultaneously $A$, already fallen into $A'$, falls into $A''$. In other words, my time moves as a whole. What is to come, which I grasp at first only through opaque shadings, comes to pass in person for me: $C_2$ `descends' into $C_1$, then gives itself in $C$ within my field of presence, and even as I meditate on this presence $C$ traces itself for my as `no longer', as my presence is in $D$. Yet if this totality is given all at once, that implies that \textit{there is no genuine problem of a unification of the series of experiences, after the fact}.}
	
\end{document}